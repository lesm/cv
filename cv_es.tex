\PassOptionsToPackage{pdfpagelabels=false}{hyperref}
\documentclass[10pt,a4paper,sans,colorlinks]{moderncv} % Font sizes: 10, 11, or 12; paper sizes: a4paper, letterpaper, a5paper, legalpaper, executivepaper or landscape; font families: sans or roman
\usepackage[utf8]{inputenc}
\usepackage[scale=0.83]{geometry} % Reduce document margins

\moderncvstyle{casual} % CV theme - options include: 'casual' (default), 'classic', 'oldstyle' and 'banking'
\moderncvcolor{orange} % CV color - options include: 'blue' (default), 'orange', 'green', 'red', 'purple', 'grey' and 'black'

%----------------------------------------------------------------------------------------
%	NAME AND CONTACT INFORMATION SECTION
%----------------------------------------------------------------------------------------

\firstname{Luis Enrique} % Your first name
\familyname{Silva Martínez} % Your last name

% All information in this block is optional, comment out any lines you don't need
%\title{Curriculum Vitae}
\address{Calle Teodoro la Rey}{Oaxaca de Juarez, Oaxaca 68033}
\address{Don Jaime 114 11, La Santa Cruz}{Santiago de Querétaro, Qro. 76020}
\mobile{951 578 75 48}
%\phone{(000) 111 1112}
%\fax{(000) 111 1113}
\email{silmar.dll@gmail.com}
%\definecolor{links}{HTML}{2A1B81}
%\hypersetup{urlcolor=links}
\homepage{lesm.github.io/}{http://lesm.github.io} % The first argument is the url for the clickable link, the second argument is the url displayed in the template - this allows special characters to be displayed such as the tilde in this example
%\extrainfo{additional information}
%\photo[70pt][0.4pt]{pictures/picture} % The first bracket is the picture height, the second is the thickness of the frame around the picture (0pt for no frame)
%\quote{"A witty and playful quotation" - John Smith}

%----------------------------------------------------------------------------------------

\begin{document}

%----------------------------------------------------------------------------------------
%	CURRICULUM VITAE
%----------------------------------------------------------------------------------------

\makecvtitle % Print the CV title

\section{Acerca de mí}
  Seguidor de las buenas prácticas de desarrollo de software desde TDD (Desarrollo guiado por pruebas), patrones de diseño, refactorización de código.
  Me gusta trabajar con equipos multidisciplinarios donde mis compañeros compartan el uso de las buenas prácticas de desarrollo de software.
%----------------------------------------------------------------------------------------
%	EDUCATION SECTION
%----------------------------------------------------------------------------------------

\section{Educación}

\cventry{2011--2018}{Ing. En Sistemas Computacionales}{Instituto Tecnológico de Oaxaca}{}{}{}  % Arguments not required can be left empty
%\cventry{2007--2010}{Bachelor of Business Studies}{The University of California}{Berkeley}{\textit{GPA -- 7.5}}{Specialized in Commerce}

%\section{Masters Thesis}

%\cvitem{Title}{\emph{Money Is The Root Of All Evil -- Or Is It?}}
%\cvitem{Supervisors}{Professor James Smith \& Associate Professor Jane Smith}
%\cvitem{Description}{This thesis explored the idea that money has been the cause of untold anguish and suffering in the world. I found that it has, in fact, not.}

%----------------------------------------------------------------------------------------
%	LANGUAGES SECTION
%----------------------------------------------------------------------------------------

\section{Idiomas}

\cvlistdoubleitem{Español (Nativo)}{Inglés, \emph{\small{buen nivel.}}}
%\cvitemwithcomment{Español}{Nativo}{}
%\cvitemwithcomment{Inglés}{Competente}{}
%\cvitemwithcomment{Dutch}{Basic}{Basic words and phrases only}

%----------------------------------------------------------------------------------------
%	WORK EXPERIENCE SECTION
%----------------------------------------------------------------------------------------
\section{Experiencia laboral}
  \cventry{Ag. 2019 -- Acte.}{Ruby on Rails Developer}{Zentiglo}{Santiago de Querétaro, Qro}{}{
    Mantenimiento y actualización del sistema principal de Zentiglo usando TDD, sistema de Facturación Electrónica.
    \newline \newline
    Actividades importantes:
    \begin{itemize}
      \item Agregando funcionalidad a la aplicación y al API de timbrado.
      \item Creación de un módulo para la generación de addendas (CFDI) con base a un XSD, usando metaprogramación.
      \item Creación de un script para timbrar facturas a traves del API de timbrado, usando incron para ejecutar el script al momento que se suba un archivo al ftp.
      \item Generación de archivos xsl para transformación de XML.
      \item Actualizando gemas, refactorización de código.
      \item Revisión de pull-request o merge-request.
    \end{itemize}}
  \cventry{Setp. 2019 -- Dic. 2019}{Ruby on Rails Developer}{Warepow}{Santiago de Querétaro, Qro}{}{
    Agregando funcionalidad a un sistema en Ruby on Rails usando TDD. Sistema para al control administrativo y financiero de obras.
  }
  \cventry{Jun. 2018 -- Jul. 2019}{Ruby on Rails Developer}{Freelance}{Oaxaca de Juárez, Oax}{}{
    Desarrollando sistemas con Ruby on Rails usando BDD y TDD.
    \newline \newline
    Actividades importantes:
    \begin{itemize}
      \item Desarrollo de un sistema para la venta en línea de libros y artículos, usando el API de Conekta (para el procesamiento de pagos).
      \item Desarrollo de un sistema para facturación electrónica de un municipio, cumpliendo con las especificaciones del SAT.
    \end{itemize}}
  \cventry{May. 2016 -- Ag. 2018}{Ruby on Rails Developer}{LogicalBricks}{Oaxaca de Juárez, Oax}{}{
    Desarrollando para Logicalbricks usando TDD, y a su vez aprendiendo de un equipo de alto desempeño.
    \newline \newline
    Mantenimiento y actualización de gemas públicas y privadas de LogicalBricks, por ejemplo:
    \begin{itemize}
      \item \textbf{fm\_timbrado\_cfdi, fm\_layout y cfdi\_parser}
        \begin{itemize}
          \item Actualización para soportar los cambios de la nómina 1.2 y la nueva facturación electrónica CFDI 3.3
            con base a las especificaciones del SAT y del PAC Facturación Moderna.
        \end{itemize}
  \end{itemize}}
  \cventry{}{}{}{}{}{
    Mantenimiento y actualización de los principales sistemas de LogicalBricks por ejemplo:
    \begin{itemize}
      \item \textbf{Infacto (Nómina y facturación)}
        \begin{itemize}
          \item Modelado y codificación de la nueva Nómina 1.2 y la nueva facturación electrónica CFDI 3.3 (SAT).
          \item A cargo del sistema agregando nuevas funcionalidades.
        \end{itemize}
      \item \textbf{IDEClara, SinPapel, Prev}
        \begin{itemize}
          \item A cargo de los sistemas agregando nuevas funcionalidades.
        \end{itemize}
    \end{itemize}
  }

%----------------------------------------------------------------------------------------
%	Workshops and Conferences
%----------------------------------------------------------------------------------------

\section{Talleres, Conferencias y Cursos}
\subsection{Asistente}
  \cventry{2014 - 2018}{Pláticas}{Ruby Oaxaca Comunidad (Oaxaca.rb)}{}{}{
    Action Cable,
    Metaprogramación en ruby,
    Matchers en RSpec,
    APIs en Rails,
    Tmux,
    Vim + Plugins,
    Vistas en Rails,
    Controladores en Rails,
    Bloques, lambdas y procs en ruby,
    Git,
    TrailBlazer,
    Middleman}
\subsection{Ponente}
  \cventry{2015}{Taller}{Introducción a Ruby}{Oaxaca de Juárez, Oax}{}{En el aniversario XII de la Carrera de Ingeniería en Sistemas Computacionales, organizado por el Instituto Tecnológico de Oaxaca.}
  \cvitem{}{Taller de 10 horas}
\subsection{Organizador}
  \cventry{2015-2017}{Taller}{Coderetreat 2015, 2016 y 2017}{Oaxaca de Juárez, Oax}{Global day of CodeRetreat}{}
  \cventry{2015-2018}{Dojos}{Ruby Oaxaca Comunidad (Oaxaca.rb)}{}{}{
    \begin{itemize}
      \item Participación en la administración de la página de la comunidad \href{http://oaxacarb.org/}{Oaxacarb.rb}.
        %\item Creador de la página \href{https://www.facebook.com/oaxacarb}{Oaxaca.rb} en Facebook \CreateHappyFace , para dar a conocer
        %la comunidad a más persona.
      \item Creador de \href{https://oaxacarb.herokuapp.com/}{oaxacarb.herokuapp} para mostrar información del canal de Slack de la comunidad.
      \item Encargado de la campaña quincenal del Coding Dojo, creando el evento en EventBrite y enviando las invitaciones a través de Mailchimp.
    \end{itemize}}
\subsection{Cursos en línea}{
  \cventry{2015-2018}{}{}{}{}{
    \begin{itemize}
      \item Ruby on Rails: An Introducction - Universidad Johns Hopkins
      \item Rails with Active Record and Action Pack - Universidad Johns Hopkins
      \item HTML, CSS and JavaScript - Universidad Científica y Tecnológica de Hong Kong
    \end{itemize}}

%----------------------------------------------------------------------------------------
%\cventry{2011--2012}{Summer Intern}{\textsc{Lehman Brothers}}{Los Angeles}{}{Rated "truly distinctive" for Analytical Skills and Teamwork.}

%------------------------------------------------

%\subsection{Miscellaneous}

%\cventry{2010--2011}{}{}{}{}{Spent some time finding myself. This was a courageous endeavour that didn't have a job title. It was quite important to my overall development though so I'm adding it to my CV. Also it explains the gap in my otherwise stellar CV.}

%\cventry{2009--2010}{Computer Repair Specialist}{Buy More}{Burbank}{}{Worked in the Nerd Herd and helped to solve computer problems. Allowed me to become expert in all forms of martial arts and weaponry.}

%----------------------------------------------------------------------------------------
%	AWARDS SECTION
%----------------------------------------------------------------------------------------

%\section{Awards}

%\cvitem{2011}{School of Business Postgraduate Scholarship}
%\cvitem{2010}{Top Achiever Award -- Commerce}

%----------------------------------------------------------------------------------------
%	COMPUTER SKILLS SECTION
%----------------------------------------------------------------------------------------

\section{Herramientas y lenguajes de Desarrollo}

\cvitem{Lenguajes}{\small{Java, Ruby, SQL, JavaScript}}
\cvitem{Frameworks}{\small{Rails (4,5), Vue(aprendiendo), Bootstrap}}
\cvitem{DBMS}{\small{MySQL, PostgreSQL}}
\cvitem{Sistemas Linux}{\small{Debian, Ubuntu, ArchLinux, Chakra}}
\cvitem{SCV}{\small{Git(Github, Bitbucket, Gitlab)}}
\cvitem{Integración continua}{\small{Semaphore CI}}
\cvitem{Otros}{\small{Vim, Tmux, RVM, Docker, Jira, Trello, Zsh, Yakuake}}

%----------------------------------------------------------------------------------------
%	COMMUNICATION SKILLS SECTION
%----------------------------------------------------------------------------------------

\section{Herramientas de Desarrollo relacionadas con Rails}

\cvitem{Pruebas}{\small{Rspect, Minitest, Capybara, Shoulda-Matchers, FactoryGirl/FactoryBot, VCR, WebMock}}
\cvitem{Vista}{\small{Simpleform, Haml, JQuery, CoffeScript}}
\cvitem{Despliegue}{\small{Capistrano, Nginx, Unicorn, Apache, Passenger, Digital Ocean, Heroku}}

%----------------------------------------------------------------------------------------
%	INTERESTS SECTION
%----------------------------------------------------------------------------------------

%\section{Interests}

%\renewcommand{\listitemsymbol}{-~} % Changes the symbol used for lists

%\cvlistdoubleitem{Piano}{Chess}
%\cvlistdoubleitem{Cooking}{Dancing}
%\cvlistitem{Running}

%----------------------------------------------------------------------------------------

\end{document}
