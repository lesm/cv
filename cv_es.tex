\PassOptionsToPackage{pdfpagelabels=false}{hyperref}
\documentclass[11pt,a4paper,sans,colorlinks]{moderncv} % Font sizes: 10, 11, or 12; paper sizes: a4paper, letterpaper, a5paper, legalpaper, executivepaper or landscape; font families: sans or roman
\usepackage[utf8]{inputenc}
\usepackage{tikz}
\usepackage{xcolor}
\usepackage{hyperref}
\usepackage[scale=0.75]{geometry} % Reduce document margins

\moderncvstyle{casual} % CV theme - options include: 'casual' (default), 'classic', 'oldstyle' and 'banking'
\definecolor{rails}{RGB}{204,0,0}

%%TeXporter:DefineEmojiColors
%\definecolor{c0a0808}{RGB}{10,8,8}
%\definecolor{cff0000}{RGB}{255,0,0}
%\definecolor{cff9700}{RGB}{255,151,0}
%\definecolor{c1a0f00}{RGB}{26,15,0}
%\definecolor{c782121}{RGB}{120,33,33}
%\definecolor{cffffff}{RGB}{255,255,255}
%\definecolor{cffefef}{RGB}{255,239,239}
%\definecolor{c250303}{RGB}{37,3,3}
%\definecolor{ce3dbdb}{RGB}{227,219,219}
%\definecolor{cd45500}{RGB}{212,85,0}
%\definecolor{cffd42a}{RGB}{255,212,42}
%\definecolor{caa4400}{RGB}{170,68,0}
%\definecolor{cff5555}{RGB}{255,85,85}
%\definecolor{cffff00}{RGB}{255,255,0}
%\definecolor{c5a5a5a}{RGB}{90,90,90}
%\definecolor{ce9c900}{RGB}{233,201,0}
%\definecolor{c2b0000}{RGB}{43,0,0}
%\definecolor{c00ffff}{RGB}{0,255,255}

%%TeXporter:EmojiHappyFace
%\def\CreateHappyFace{
  %\begin{tikzpicture}[scale=0.15]
    %\pgfdeclareradialshading{ballshading}{\pgfpoint{-10bp}{10bp}}
    %{color(0bp)=(yellow!40!white);
    %color(9bp)=(yellow!55!white);
    %color(18bp)=(yellow!60!orange);
    %color(25bp)=(yellow!50!orange);
    %color(50bp)=(black)}
    %\pgfpathcircle{\pgfpoint{0cm}{0cm}}{2cm}
    %\pgfshadepath{ballshading}{20}
    %\pgfusepath{}
    %\pgfdeclareradialshading{ballshading}{\pgfpoint{-10bp}{10bp}}
    %{color(0bp)=(yellow!40!white);
    %color(9bp)=(yellow!30!white);
    %color(18bp)=(yellow!60!orange);
    %color(25bp)=(yellow!50!orange);
    %color(50bp)=(orange)}
    %\pgfpathcircle{\pgfpoint{0cm}{0cm}}{1.85cm}
    %\pgfshadepath{ballshading}{20}
    %\pgfusepath{}
    %%Head hat
    %\path[yshift=138,xshift=-140,scale=0.012,yscale=-1, fill=white,opacity=0.234] (411.4196,246.6473) .. controls
    %(344.7661,246.6473) and (288.7870,289.5501) .. (273.0134,347.5848) .. controls
    %(311.3176,336.0898) and (355.6292,329.4911) .. (402.8571,329.4911) .. controls
    %(457.9662,329.4911) and (509.1043,338.4615) .. (551.3571,353.7411) .. controls
    %(537.9844,292.6125) and (480.4076,246.6473) .. (411.4196,246.6473) -- cycle;
    %%Left eye
    %\path[yshift=163,xshift=-68,scale=0.012,inner color=c782121!70!yellow,outer color=c782121!60!black,yscale=-1] (157.4105,420.0859) .. controls (145.5758,420.0859) and
    %(135.9730,441.1746) .. (135.9730,467.2109) .. controls (135.9730,472.5231) and
    %(136.3929,477.6340) .. (137.1293,482.3984) .. controls (139.9795,463.6457) and
    %(148.6651,449.9922) .. (158.9730,449.9922) .. controls (167.3672,449.9922) and
    %(174.7174,459.0360) .. (178.6918,472.5234) .. controls (178.7809,470.7772) and
    %(178.8168,469.0096) .. (178.8168,467.2109) .. controls (178.8168,441.1746) and
    %(169.2452,420.0859) .. (157.4105,420.0859) -- cycle;
    %%Right eye
    %\path[yshift=163,xshift=68,xscale=-1,scale=0.012,inner color=c782121!70!yellow,outer color=c782121!60!black,yscale=-1] (157.4105,420.0859) .. controls (145.5758,420.0859) and
    %(135.9730,441.1746) .. (135.9730,467.2109) .. controls (135.9730,472.5231) and
    %(136.3929,477.6340) .. (137.1293,482.3984) .. controls (139.9795,463.6457) and
    %(148.6651,449.9922) .. (158.9730,449.9922) .. controls (167.3672,449.9922) and
    %(174.7174,459.0360) .. (178.6918,472.5234) .. controls (178.7809,470.7772) and
    %(178.8168,469.0096) .. (178.8168,467.2109) .. controls (178.8168,441.1746) and
    %(169.2452,420.0859) .. (157.4105,420.0859) -- cycle;
    %%mouth
    %\path[yshift=203,xshift=-90,scale=0.024,yscale=-1,draw=black,inner color=c782121!70!yellow,outer color=c782121!60!black,miter limit=4.00,line width=0.820pt]
    %(83.2188,316.4062) .. controls (86.0968,339.8625) and (107.1942,358.0625) ..
    %(132.8438,358.0625) .. controls (158.3006,358.0625) and (179.3095,340.1386) ..
    %(182.4375,316.9375) .. controls (167.0177,319.5504) and (150.8782,320.9375) ..
    %(134.2812,320.9375) .. controls (116.6268,320.9375) and (99.5245,319.3512) ..
    %(83.2188,316.4062) -- cycle;
    %%teeth
    %\path[yshift=203,xshift=-90,scale=0.024,yscale=-1,inner color=white,outer color=ce3dbdb!80!black,draw=black,line width=0.6] (83.3970,316.4171) .. controls (84.1788,322.7888) and
    %(86.2941,328.7681) .. (89.4908,334.1046) .. controls (103.3990,334.8620) and
    %(119.1174,338.6714) .. (133.0437,338.3979) .. controls (148.5504,338.0933) and
    %(162.2806,333.5914) .. (177.7095,332.0108) .. controls (180.1799,327.3657) and
    %(181.8923,322.3146) .. (182.6157,316.9483) .. controls (167.1960,319.5612) and
    %(151.0564,320.9483) .. (134.4595,320.9483) .. controls (116.8050,320.9483) and
    %(99.7027,319.3620) .. (83.3970,316.4171) -- cycle;
    %%tangue
    %\path[yshift=203,xshift=-110,scale=0.024,yscale=-1,xscale=1.25,inner color=cff5555!80!black,outer color=c782121!60!black,opacity=0.9] (115.5196,352.1215) .. controls (111.3671,350.2285) and
    %(105.6834,345.8852) .. (102.8892,342.4698) -- (97.8089,336.2598) --
    %(102.2146,336.3376) .. controls (104.6377,336.3804) and (111.6672,337.1200) ..
    %(117.8358,337.9812) .. controls (124.3649,338.8926) and (133.4252,338.8866) ..
    %(139.5191,337.9667) .. controls (145.2764,337.0975) and (153.3516,335.9044) ..
    %(157.4639,335.3152) -- (164.9409,334.2440) -- (157.7177,342.3122) .. controls
    %(146.0180,355.3804) and (130.5538,358.9753) .. (115.5196,352.1216) -- cycle;
  %\end{tikzpicture}
  %}
  %%TeXporter:EndEmojiHappyFace

\moderncvcolor{grey} % CV color - options include: 'blue' (default), 'orange', 'green', 'red', 'purple', 'grey' and 'black'

%----------------------------------------------------------------------------------------
%	NAME AND CONTACT INFORMATION SECTION
%----------------------------------------------------------------------------------------

\firstname{Luis Enrique} % Your first name
\familyname{Silva Martínez} % Your last name

% All information in this block is optional, comment out any lines you don't need
%\title{Curriculum Vitae}
\address{Calle Teodoro la Rey}{Oaxaca de Juarez, Oaxaca 68033}
\mobile{951 578 75 48}
%\phone{(000) 111 1112}
%\fax{(000) 111 1113}
\email{silmar.dll@gmail.com}
%\definecolor{links}{HTML}{2A1B81}
%\hypersetup{urlcolor=links}
\homepage{lesm.github.io/}{http://lesm.github.io} % The first argument is the url for the clickable link, the second argument is the url displayed in the template - this allows special characters to be displayed such as the tilde in this example
%\extrainfo{additional information}
%\photo[70pt][0.4pt]{pictures/picture} % The first bracket is the picture height, the second is the thickness of the frame around the picture (0pt for no frame)
%\quote{"A witty and playful quotation" - John Smith}

%----------------------------------------------------------------------------------------

\begin{document}

%----------------------------------------------------------------------------------------
%	COVER LETTER
%----------------------------------------------------------------------------------------

% To remove the cover letter, comment out this entire block

%\clearpage

%\recipient{HR Department}{Corporation\\123 Pleasant Lane\\12345 City, State} % Letter recipient
%\date{\today} % Letter date
%\opening{Dear Sir or Madam,} % Opening greeting
%\closing{Sincerely yours,} % Closing phrase
%\enclosure[Attached]{curriculum vit\ae{}} % List of enclosed documents

%\makelettertitle % Print letter title

%\lipsum[1-2] % Dummy text
%\lipsum[4] % Dummy text

%\makeletterclosing % Print letter signature

\newpage

%----------------------------------------------------------------------------------------
%	CURRICULUM VITAE
%----------------------------------------------------------------------------------------

\makecvtitle % Print the CV title

%----------------------------------------------------------------------------------------
%	EDUCATION SECTION
%----------------------------------------------------------------------------------------

\section{Educación}

\cventry{2011--2018}{Ingeniero En Sistemas Computacionales}{Instituto Tecnológico de Oaxaca, Oaxaca, México}{ITO}{}{}  % Arguments not required can be left empty
%\cventry{2007--2010}{Bachelor of Business Studies}{The University of California}{Berkeley}{\textit{GPA -- 7.5}}{Specialized in Commerce}

%\section{Masters Thesis}

%\cvitem{Title}{\emph{Money Is The Root Of All Evil -- Or Is It?}}
%\cvitem{Supervisors}{Professor James Smith \& Associate Professor Jane Smith}
%\cvitem{Description}{This thesis explored the idea that money has been the cause of untold anguish and suffering in the world. I found that it has, in fact, not.}

%----------------------------------------------------------------------------------------
%	LANGUAGES SECTION
%----------------------------------------------------------------------------------------

\section{Idiomas}

\cvitemwithcomment{Español}{Nativo}{}
\cvitemwithcomment{Inglés}{Competente}{Buen nivel conversacional (Nivel B1)}
%\cvitemwithcomment{Dutch}{Basic}{Basic words and phrases only}

%----------------------------------------------------------------------------------------
%	WORK EXPERIENCE SECTION
%----------------------------------------------------------------------------------------

\section{Experiencia laboral}
  \cventry{Jun 2018 -- Actualmente}{Freelance}{}{Oaxaca, México}{}{Ruby on Rails Developer
    \newline \newline
    Actividades importantes:
    \begin{itemize}
      \item Planeación, modelado y construcción de un sistema para la venta en línea de libros y artículos, usando el API de Conekta (para el procesamiento de pagos). Desarrollado en Rails usando TDD.
      \item Planeación, modelado y construcción de un sistema para facturación electrónica de un municipio, cumpliendo con las especificaciones del SAT, Desarrollado en Rails usando BDD y TDD.
    \end{itemize}}

  \cventry{Mayo 2016 -- Agosto 2018}{Ruby on Rails Developer}{\textsc{LogicalBricks}}{Oaxaca, México}{}{ Mantenimiento y actualización de los principales sistemas de Logicalbricks usando TDD, y a su vez aprendiendo de un equipo de alto desempeño.
  \newline \newline
  Actividades importantes:
  \begin{itemize}
    \item Manteniendo y actualizando gemas publicas y privadas de LogicalBricks, por ejemplo:
      \begin{itemize}
        \item \textbf{fm\_timbrado\_cfdi}
          \begin{itemize}
            \item Actualización para soportar los cambios de la versión del comprobante fiscal digital por internet (CFDI 3.3),
              con base a las especificaciones del PAC Facturación Moderna.
          \end{itemize}
        \item \textbf{fm\_layout}
          \begin{itemize}
            \item Actualización para soportar los cambios de la Nómina 1.2,
              con base a las especificaciones del PAC Facturación Moderna.
            \item Actualización para soportar los cambios de la versión del comprobante fiscal digital por internet (CFDI 3.3),
              con base a las especificaciones del PAC Facturación Moderna.
          \end{itemize}
        \item \textbf{cfdi\_parser}
          \begin{itemize}
            \item Actualización para soportar los cambios de la versión del comprobante fiscal digital por internet (CFDI 3.3)
          \end{itemize}
      \end{itemize}
  \end{itemize}}

  \cventry{}{}{}{}{}{
    \begin{itemize}
      \item Manteniendo y actualizando los principales sistemas de LogicalBricks por ejemplo:
        \begin{itemize}
          \item \textbf{Infacto (Nómina y facturación)}
            \begin{itemize}
              \item Modelado y codificación de la nueva Nómina 1.2, cumpliendo con las especificaciones del SAT.
              \item Modelado y codificación de la nueva facturación electrónica (CFDI 3.3), cumpliendo con las especificaciones del SAT.
              \item A cargo del sistema agregando nuevas funcionalidades y resolviendo errores.
            \end{itemize}
          \item \textbf{IDEClara (Para reportar el IDE)}
            \begin{itemize}
              \item A cargo del sistema agregando nuevas funcionalidades y resolviendo errores.
            \end{itemize}
          \item \textbf{SinPapel (Para descargar facturas automáticamente)}
            \begin{itemize}
              \item A cargo del sistema agregando nuevas funcionalidades y resolviendo errores.
            \end{itemize}
          \item \textbf{Prev(Para control de previsión social)}
            \begin{itemize}
              \item Agregando cambios y resolviendo errores.
            \end{itemize}
        \end{itemize}
    \end{itemize}
  }

%----------------------------------------------------------------------------------------
%	Workshops and Conferences
%----------------------------------------------------------------------------------------

\section{Talleres, Conferencias y Cursos}
\subsection{Asistente}
  \cventry{2018}{Action Cable}{Plática}{Ruby Oaxaca Comunidad (Oaxaca.rb)}{}{Introducción Action Cable por Azarel Doroteo}
  \cventry{2017}{Metaprogramación en Ruby}{Plática}{Ruby Oaxaca Comunidad (Oaxaca.rb)}{}{Metaprogramación en Ruby por Azarel Doroteo}
  \cventry{}{Matchers en Rspec}{Plática}{Ruby Oaxaca Comunidad (Oaxaca.rb)}{}{Matchers personalizados en Rspec por Azarel Doroteo}
  \cventry{2016}{APIs en Rails}{Plática}{Ruby Oaxaca Comunidad (Oaxaca.rb)}{}{Introducción APIs en Rails por Fernando Villalobos}
  \cventry{}{Tmux}{Plática}{Ruby Oaxaca Comunidad (Oaxaca.rb)}{}{Tmux por Hermes Ojeda}
  \cventry{}{Vim + Plugins}{Plática}{Ruby Oaxaca Comunidad (Oaxaca.rb)}{}{Vim + Plugins para trabajar con Rails por Azarel Doroteo}
  \cventry{}{Vistas en Rails}{Plática}{Ruby Oaxaca Comunidad (Oaxaca.rb)}{}{Vistas en Rails por Fernando Villalobos}
  \cventry{2015}{Controladores en Rails}{Plática}{Ruby Oaxaca Comunidad (Oaxaca.rb)}{}{Controladores en Rails por Hermes Ojeda}
  \cventry{}{Bloques, Lambdas y Procs en Ruby}{Plática}{Ruby Oaxaca Comunidad (Oaxaca.rb)}{}{Introducción a Bloques, Lambdas y Procs en Ruby por Azarel Doroteo}
  \cventry{}{Git}{Plática}{Ruby Oaxaca Comunidad (Oaxaca.rb)}{}{Introducción a Git por Hermes Ojeda}
  \cventry{}{Rspec-Given}{Plática}{Ruby Oaxaca Comunidad (Oaxaca.rb)}{}{Introducción a Rspect-Given por Azarel Doroteo}
  \cventry{}{TrailBlazer}{Plática}{Ruby Oaxaca Comunidad (Oaxaca.rb)}{}{Conociendo TrailBlazer por Nick Sutterer}
  \cventry{2014}{Middleman}{Taller}{Ruby Oaxaca Comunidad (Oaxaca.rb)}{}{Cómo hacer páginas estáticas con Ruby por Fernando Villalobos}
\subsection{Ponente}
  \cventry{2015}{Introducción a Ruby}{Taller}{Oaxaca de Juárez, Oaxaca, México}{}{En el aniversario XII de la Carrera de Ingeniería en Sistemas Computacionales, organizado por el Instituto Tecnológico de Oaxaca.}
  \cvitem{}{Taller de 10 horas}
\subsection{Organizador}
  \cventry{2015-2017}{Coderetreat 2015, 2016 y 2017}{Taller}{Oaxaca de Juárez, Oaxaca, México}{Global day of CodeRetreat}{}
  \cventry{2015-2018}{Oaxaca.rb}{}{Comunidad de programadores en ruby}{}{}
  \cvitem{}{
    \begin{itemize}
      \item Participación en la administración de la página de la comunidad \href{http://oaxacarb.org/}{Oaxacarb.rb}
      %\item Creador de la página \href{https://www.facebook.com/oaxacarb}{Oaxaca.rb} en Facebook \CreateHappyFace , para dar a conocer
          %la comunidad a más persona.
      \item Creador de \href{https://oaxacarb.herokuapp.com/}{oaxacarb.herokuapp} para mostrar información del canal de Slack de la comunidad.
      \item Encargado de la campaña quincenal del Coding Dojo, creando el evento en EventBrite y enviando las invitaciones a través de Mailchimp.
    \end{itemize}}
\subsection{Cursos en línea}{
  \cvitem{2015}{ Coursera
    \begin{itemize}
      \item Ruby on Rails: An Introducction - Universidad Johns Hopkins
      \item Rails with Active Record and Action Pack - Universidad Johns Hopkins
      \item HTML, CSS and JavaScript - Universidad Científica y Tecnológica de Hong Kong
    \end{itemize}}

%----------------------------------------------------------------------------------------
%\cventry{2011--2012}{Summer Intern}{\textsc{Lehman Brothers}}{Los Angeles}{}{Rated "truly distinctive" for Analytical Skills and Teamwork.}

%------------------------------------------------

%\subsection{Miscellaneous}

%\cventry{2010--2011}{}{}{}{}{Spent some time finding myself. This was a courageous endeavour that didn't have a job title. It was quite important to my overall development though so I'm adding it to my CV. Also it explains the gap in my otherwise stellar CV.}

%\cventry{2009--2010}{Computer Repair Specialist}{Buy More}{Burbank}{}{Worked in the Nerd Herd and helped to solve computer problems. Allowed me to become expert in all forms of martial arts and weaponry.}

%----------------------------------------------------------------------------------------
%	AWARDS SECTION
%----------------------------------------------------------------------------------------

%\section{Awards}

%\cvitem{2011}{School of Business Postgraduate Scholarship}
%\cvitem{2010}{Top Achiever Award -- Commerce}

%----------------------------------------------------------------------------------------
%	COMPUTER SKILLS SECTION
%----------------------------------------------------------------------------------------

\section{Herramientas y lenguajes de Desarrollo}

\cvitem{Lenguajes}{Java(intermedio), Ruby(intermedio), SQL(intermedio), JavaScript(intermedio)}
\cvitem{Frameworks}{Rails 4,5 (intermedio) Vue(aprendiendo), Bootstrap(intermedio)}
\cvitem{DBMS}{MySQL, PostgreSQL}
\cvitem{Sistemas Linux}{Debian, Ubuntu, ArchLinux, Chakra}
\cvitem{SCV}{Git(Github, Bitbucket)}
\cvitem{Integración continua}{Semaphore CI}
\cvitem{Otros}{Vim(avanzado), Tmux, Heroku, RVM}

%----------------------------------------------------------------------------------------
%	COMMUNICATION SKILLS SECTION
%----------------------------------------------------------------------------------------

\section{Herramientas de Desarrollo relacionadas con Rails}

\cvitem{Pruebas}{Rspect, Minitest, Capybara, Shoulda-Matchers, FactoryGirl/FactoryBot}
\cvitem{Vista}{Simpleform, Haml, JQuery, CoffeScript}
\cvitem{Despliegue}{Capistrano, Nginx, Unicorn, Apache, Passenger, Digital Ocean}

%----------------------------------------------------------------------------------------
%	INTERESTS SECTION
%----------------------------------------------------------------------------------------

%\section{Interests}

%\renewcommand{\listitemsymbol}{-~} % Changes the symbol used for lists

%\cvlistdoubleitem{Piano}{Chess}
%\cvlistdoubleitem{Cooking}{Dancing}
%\cvlistitem{Running}

%----------------------------------------------------------------------------------------

\end{document}
